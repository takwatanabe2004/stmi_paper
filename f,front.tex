%=========================================================================%
% Paper title
%=========================================================================%
\renewcommand{\HSPACE} {\hspace{-1.9pt}}
\newcommand{\MYTITLE}{\hspace{-5pt}Multisite Disease Classification with Functional \\ \hspace{-15pt} Connectomes \HSPACE via \HSPACE Multitask \HSPACE Structured \HSPACE Sparse \HSPACE SVM}

\title{\MYTITLE}

\newcommand{\MYTITLERUN}{Multisite Disease Classification via Multitask Structured Sparse SVM}
\titlerunning{{\MYTITLERUN}}

%=========================================================================%
% the name(s) of the author(s) follow(s) next
%=========================================================================%
\author{Takanori Watanabe$^1$, Daniel Kessler$^2$, Clayton Scott$^1$, Chandra Sripada$^2$}
\authorrunning{{T. Watanabe, D. Kessler, C. Scott, C. Sripada}} 
\institute{$^1$Department of EECS, $^2$Department of Psychiatry \\ University of Michigan, Ann Arbor, MI 48109, USA\\
\mailsa\\}
\urldef{\mailsa}\path|{takanori, kesslerd, clayscot, csripada}@umich.edu|
\maketitle

%=========================================================================%
% abstract
%=========================================================================%
\begin{abstract}
There is great interest in developing imaging-based methods for diagnosing neuropsychiatric conditions. 
To this end, multiple data-sharing initiatives have been launched in the neuroimaging field, where datasets are collected across multiple imaging sites.
While this enables researchers to study the disorders of interest with substantial sample size, it also creates new challenges since the data aggregation process introduces various sources of \mbox{site-specific} heterogeneities.
To address this issue, we introduce a multitask structured sparse support vector machine (SVM) that uses resting state functional connectomes (FCs) as the features for predicting diagnostic labels.
Specifically, we employ a penalty that accounts for the following two-way structure that exists in a multisite FC dataset: (1)~the $6$-D \emph{spatial structure} in the FCs captured via either the GraphNet, fused Lasso, or the isotropic total variation penalty, and (2) the \emph{inter-site} structure captured via the multitask \MTL-penalty.
To solve the resulting high dimensional optimization problem, we introduce an extension to a recently proposed algorithm based on the alternating direction method.
The potential utility of the proposed method is demonstrated on the multisite ADHD-200 dataset. 
\keywords{Multitask learning, structured sparsity, support vector machine, resting-state fMRI, alternating direction method}
\end{abstract}

\setlength{\abovedisplayskip}{4.5pt}
\setlength{\belowdisplayskip}{6pt}
